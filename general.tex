% GENERAL UTILITIES ============================================================

% Left/Right -------------------------------------------------------------------
% These two-letter command names mostly come from old personal conventions.

\newcommand{\pr}[1]{\left( #1 \right)} % PaRentheses
    \newcommand{\tuple}{\pr}
        \newcommand{\pair}[2]{\tuple{ #1 , #2 }}
        \newcommand{\triple}[3]{\tuple{ #1 , #2 , #3 }}
\newcommand{\sq}[1]{\left[ #1 \right]} % SQuare brackets
    \newcommand{\arr}{\sq}
\newcommand{\cb}[1]{\left\lbrace #1 \right\rbrace} % Curly Braces
    \newcommand{\set}{\cb}
\newcommand{\vv}[1]{\left\lvert #1 \right\rvert} % single pipes (vert vert)
    \newcommand{\abs}{\vv}
    \newcommand{\size}{\vv}
\newcommand{\VV}[1]{\left\lVert #1 \right\rVert} % double pipes (Vert Vert)
    \newcommand{\norm}{\VV}
    \newcommand{\length}{\VV}
\newcommand{\ag}[1]{\left\langle #1 \right\rangle} % AnGle brackets
\newcommand{\floor}[1]{\left\lfloor #1 \right\rfloor} % floor function
\newcommand{\ceiling}[1]{\left\lceil #1 \right\rceil} % ceiling function
    \newcommand{\ceil}{\ceiling}

% Left Scripts -----------------------------------------------------------------

\newcommand{\lsub}[2]{{\vphantom{ #1 }}_{ #2 } #1} % sub
\newcommand{\lsup}[2]{{\vphantom{ #1 }}^{ #2 } #1} % super
\newcommand{\lsubp}[3]{{\vphantom{ #1 }}_{ #2 }^{ #3 } #1} % sub, super
\newcommand{\lsupb}[3]{{\vphantom{ #1 }}^{ #2 }_{ #3 } #1} % super, sub

% Functions --------------------------------------------------------------------
% The macros here are mainly used to define other macros.

\newcommand{\op}{\operatorname}

% \mathopen and \mathclose reduce padding around Left/Right blocks.
\newcommand{\prarg}[1]{\mathopen{} \pr{ #1 } \mathclose{}}
\newcommand{\sqarg}[1]{\mathopen{} \sq{ #1 } \mathclose{}}
\newcommand{\cbarg}[1]{\mathopen{} \cb{ #1 } \mathclose{}}

% functions
\newcommand{\fn}[1]{{ #1 }\,}
\newcommand{\fnpr}[1]{{ #1 }\prarg}
\newcommand{\fnsq}[1]{{ #1 }\sqarg}
\newcommand{\fncb}[1]{{ #1 }\sqarg}

% Array Utilities --------------------------------------------------------------

\newcommand{\range}[2]{{ #1 }\,{..}\,{ #2 }}
\newcommand{\idx}[1]{\mathopen{} \sq{ #1 } \mathclose{}}

% Environments -----------------------------------------------------------------

% align
\newcommand{\ALIGN}[1]{\begin{aligned} #1 \end{aligned}}
\newcommand{\SINCE}[1]{\text{since \( #1 \)}}
\newcommand{\BECAUSE}[1]{\text{because \( #1 \)}}
    \newcommand{\BC}{\BECAUSE}

% cases
\newcommand{\CASES}[1]{\begin{cases} #1 \end{cases}}
\newcommand{\IF}[1]{\text{if \( #1 \)}}
\newcommand{\WHEN}[1]{\text{when \( #1 \)}}
\newcommand{\OTHERWISE}{\text{otherwise}}
    \newcommand{\OW}{\OTHERWISE}

% matrix/vector
\newcommand{\mat}[1]{\begin{bmatrix} #1 \end{bmatrix}}
    \newcommand{\vect}{\mat}

% Slash Fraction ---------------------------------------------------------------

\newcommand{\flac}[2]{\left. #1 \middle/ #2 \right.}

% Miscellaneous ----------------------------------------------------------------

\newcommand{\defequals}{\mathrel{\mathop:}=}
    \newcommand{\defeq}{\defequals}
    \newcommand{\deq}{\defequals}
\newcommand{\divides}{\mid}


% SETS =========================================================================

% Set Builder Notation ---------------------------------------------------------

\newcommand{\suchthat}{\ \middle|\ }
    \newcommand{\where}{\suchthat}

% Common Sets ------------------------------------------------------------------

\newcommand{\Naturals}{\mathbb{N}}
    \newcommand{\Nats}{\Naturals}
    \newcommand{\N}{\Naturals}
\newcommand{\Integers}{\mathbb{Z}}
    \newcommand{\Ints}{\Integers}
    \newcommand{\Z}{\Integers}
\newcommand{\Reals}{\mathbb{R}}
    \newcommand{\R}{\Reals}
\newcommand{\Complex}{\mathbb{C}}
    \newcommand{\C}{\Complex}
\newcommand{\Polynomials}[2]{\fnsubpr{P}{ #1 }{ #2 }}
    \newcommand{\Nomials}{\Polynomials}
\newcommand{\Matrices}[3]{\fnsubpr{M}_{ #1 , #2 }{ #3 }}
    \newcommand{\Mats}{\Matrices}

% Intervals --------------------------------------------------------------------

\newcommand{\cc}[2]{\left[ #1 , #2 \right]} % Closed-Closed
\newcommand{\co}[2]{\left[ #1 , #2 \right)} % Closed-Open
\newcommand{\oc}[2]{\left( #1 , #2 \right]} % Open-Closed
\newcommand{\oo}[2]{\left( #1 , #2 \right)} % Open-Open

% Set Operations ---------------------------------------------------------------

\newcommand{\union}{\cup}
\newcommand{\Union}{\bigcup}
\newcommand{\intersect}{\cap}
\newcommand{\Intersect}{\bigcap}
\newcommand{\directsum}{\oplus}
    \newcommand{\dsum}{\directsum}


% CALCULUS =====================================================================

% Derivative notations ---------------------------------------------------------

\newcommand{\dd}[3][]{\frac{d^{ #1 } { #2 }}{{d { #3 }}^{ #1 }}}
\newcommand{\pdd}[3][]{\frac{\partial^{ #1 } { #2 }}{{\partial { #3 }}^{ #1 }}}
\newcommand{\prm}{^\prime}
\newcommand{\pprm}{^{\prime\prime}}

% At Notation ------------------------------------------------------------------

\newcommand{\at}[2]{\left. { #2 } \right\rvert_{ #1 }}
\newcommand{\between}[3]{\left. { #3 } \right\rvert_{ #1 }^{ #2 }}


% NAMED FUNCTIONS ==============================================================
% The suffix p refers to the parenthesized argument.

% Utilities --------------------------------------------------------------------

% minimum
\newcommand{\minp}{\fnpr{\min}}
\newcommand{\mins}{\fnsq{\min}}
\newcommand{\minc}{\fncb{\min}}

% maximum
\newcommand{\maxp}{\fnpr{\max}}
\newcommand{\maxs}{\fnsq{\max}}
\newcommand{\maxc}{\fncb{\max}}

% Number Theory ----------------------------------------------------------------

\newcommand{\lcmp}{\fnpr{\lcm}}
\newcommand{\gcdp}{\fnpr{\gcd}}

% Logarithms and Exponentiation ------------------------------------------------

\newcommand{\expp}{\fnpr{\exp}}
\newcommand{\lnp}{\fnpr{\ln}}
\newcommand{\logp}[1][]{\fnpr{\log_{ #1 }}}

% Trigonometry -----------------------------------------------------------------

\newcommand{\cosp}[1][]{\fnpr{\cos^{ #1 }}}
\newcommand{\sinp}[1][]{\fnpr{\sin^{ #1 }}}
\newcommand{\tanp}[1][]{\fnpr{\tan^{ #1 }}}
\newcommand{\secp}[1][]{\fnpr{\sec^{ #1 }}}
\newcommand{\cscp}[1][]{\fnpr{\csc^{ #1 }}}
\newcommand{\cotp}[1][]{\fnpr{\cot^{ #1 }}}
