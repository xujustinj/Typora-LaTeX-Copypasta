% FLOATING POINT ===============================================================

% Notation ---------------------------------------------------------------------

\newcommand{\float}{\fnpr{fl}}
    \newcommand{\fl}{\float}
\newcommand{\recur}[1]{\overline{ #1 }} % recurring digits in a decimal number
    % \repeat would also be a good name for this, if it weren't reserved

% Operators --------------------------------------------------------------------

\newcommand{\fplus}{\oplus}
    \newcommand{\fadd}{\fplus}
\newcommand{\fminus}{\ominus}
    \newcommand{\fsubtract}{\fminus}
    \newcommand{\fsub}{\fsub}
\newcommand{\ftimes}{\otimes}
    \newcommand{\fmultiply}{\ftimes}
        \newcommand{\fmult}{\fmultiply}
        \newcommand{\fmul}{\fmultiply}
    \newcommand{\fproduct}{\ftimes}
        \newcommand{\fprod}{\fproduct}
\newcommand{\fdivide}{\oslash}
    \newcommand{\fdiv}{\fdivide}

% Numerical Methods ------------------------------------------------------------

\newcommand{\emach}{\epsilon_{\mathit{mach}}} % machine epsilon
\newcommand{\LocalTruncationError}{{LTE}}
    \newcommand{\LTE}{\LocalTruncationError}
